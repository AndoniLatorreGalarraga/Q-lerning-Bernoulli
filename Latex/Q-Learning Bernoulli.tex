%%% Compile with pdfLaTeX
\documentclass[a4paper,11pt]{article}

\usepackage{graphicx}
\usepackage{natbib}
\usepackage{subcaption}
%\usepackage{subfig}

\usepackage{algorithmic}
\usepackage{algorithm}
\usepackage{tikz}
\usepackage{amsfonts}
\usetikzlibrary{positioning}
\pagenumbering{gobble}
%\doublespacing %\singlespacing %\onehalfspacing is default
\usepackage[margin=1in]{geometry}
\usepackage{hyperref}

\usepackage{amsmath}

\usepackage{mathtools}
\newcommand{\nota}[3][2ex]{
    \underset{\mathclap{
        \begin{tikzpicture}
          \draw[->] (0, 0) to ++(0,#1);
          \node[below] at (0,0) {#3};
        \end{tikzpicture}}}{#2}
}
%\usepackage[basque]{babel}

\title{Q-learning for the Optimal Bernoulli Routing}

\begin{document}

\author{}
\date{}

\maketitle

\section*{Model Description}

We consider a system with $N$ parallel queues and a single dispatcher (or router). 
Jobs arrive to the system according to a Poisson process of rate $\lambda$. We assume
that the service time of jobs in the queues is exponentially distributed and we denote by
$r_i$ the rate at which jobs at queue $i$ are served. When a job arrives to the dispatcher
it is immediately routed to Queue $i$ with probability $p_i$. Hence, $\sum_{i=1}^Np_i=1$. We aim to study the value of the routing probabilities such that the mean number of customers is minimized.
 
The authors in \cite{AAP11} study this system and 
characterize the optimal routing probability. Here, aim to show that Q-learning can 
be used to learn which is the optimal routing probability. 

\section*{Markov Decision Process Formulation}

We formulate the above problem as a Markov Decision Process in discrete time in which 
the discretization is carried out when a job arrives to the system. We consider the discounted cost problem, that is, we aim to find the probabilities $p_1,p_2,\dots,p_N$ such that the following expression is minimized:
 \begin{equation}
\mathbb E\left[\sum_{t=0}^\infty \sum_{i=1}^N\delta^t Q_i(t))\right],
\end{equation}
where $\delta\in(0,1)$ and $Q_i(t)$ is the number of jobs in Queue $i$ at time slot $t$.

Let us define the following elements of the Markov Decision Process we consider:
\begin{itemize}
\item The state represents the number of jobs in each queue. Therefore, the set of states is a vector of size $N$ such that each element is $0$ or a natural number. That is, $\mathcal S=\mathbb N_0\times \dots \times \mathbb N_0$, where $\mathbb N_0=\mathbb N \cup \{0\}$.
\item The action is a vector $(p_1,\dots,p_N)$ such that the i-th element is the probability that an incoming job is sent to Queue $i$. Therefore, the set of actions is a probability vector of size $N$. We assume that each element of the probability vector belongs to $\{0,\frac{1}{d},\frac{2}{d},\dots,\frac{d-1}{d},1,\}$ for a fixed $d$ (this means that the probabilities will never be real values).
\item The cost is the total number of customers in the system when a job is sent to one of the queues.
\item The transition probabilities. Between two arrivals, one or more jobs can be served at Queue $i$. We denote by $q_{i,j}$ the probability that
$j$ jobs are served at Queue $i$ in a interval of time of $\lambda$. 
\end{itemize}

\section*{Calculation of the transition probabilities}
The probabilities for a $G/M/1$ queue are calculated in \cite{ivo}. In our case:
$$
q_{i,j} = 
\left\{\begin{array}{ll}
    \displaystyle \int_0^\infty \frac{(r_i t)^j}{j!} e^{-r_i t} \lambda e^{-\lambda t} dt & j < Q_i \\
    \displaystyle 1- \sum_{k=0}^{Q_i -1} q_{i,k} & j = Q_i \\
\end{array}\right.
$$
Whre $Q_i$ is the number of jons at queue $i$. Thus,
$$
q_{i,0} = \int_0^\infty \lambda e^{-t(r_i + \lambda)} dt
=
\frac{\lambda}{r_i + \lambda}
$$
When $0<j<Q_i$, integrating by parts we get:
$$
\int_0^\infty \frac{(r_i t)^j}{j!} e^{-r_i t} \lambda e^{-\lambda t} dt
=
\frac{r_i^j}{j!} \lambda \int_0^\infty t^j e^{-t(r_i + \lambda)} dt
\nota{=}{$\boxed{\begin{array}{cc}
    u = t^j & dv = e^{-t(r_i+\lambda)} \\
    du = jt^{j-1} & v = \frac{- e^{-t(r_i+\lambda)}}{r_i+\lambda} dt \\
\end{array}}$}
\frac{r_i^j}{j!} \lambda \left( \left. \frac{-t^j e^{-t(r_i+\lambda)}}{r_i+\lambda} \right|_0^\infty + \int_0^\infty jt^{j-1} \frac{e^{-t(r_i+\lambda)}}{r_i+\lambda} dt \right) = 
$$
$$
=
\frac{r_i^j}{j!} \lambda \int_0^\infty jt^{j-1} \frac{e^{-t(r_i+\lambda)}}{r_i+\lambda} dt
=
\frac{r_i}{r_i + \lambda} \int_0^\infty \frac{(r_i t)^{j-1}}{(j-1)!} \lambda e^{-t(r_i + \lambda)} dt
=
\frac{r_i}{r_i + \lambda} q_{i,j-1}
=
$$
$$
=
\left( \frac{r_i}{r_i + \lambda} \right)^j q_{i,0}
=
\frac{r_i^j \lambda}{(r_i + \lambda)^{j+1}}
$$
When $j = Q_i$:
$$
q_{i, j}
=
1 - \sum_{k=0}^{Q_i -1} q_{i,k}
=
1 - \sum_{k=0}^{Q_i -1} \frac{r_i^k \lambda}{(r_i + \lambda)^{k+1}}
=
1 - \frac{\lambda}{r_i + \lambda} \sum_{k=0}^{Q_i -1} \left( \frac{r_i}{r_i + \lambda} \right)^k
=
1 - \frac{\lambda}{r_i + \lambda} \left( \frac{1- \left( \frac{r_i}{r_i + \lambda} \right)^{Q_i} }{1-\left( \frac{r_i}{r_i + \lambda} \right)} \right)
=
$$
$$
=
1 - \frac{\lambda}{r_i + \lambda} \left( \frac{1- \left( \frac{r_i}{r_i + \lambda} \right)^{Q_i}}{\frac{\lambda}{r_i + \lambda}} \right)
=
\left( \frac{r_i}{r_i + \lambda} \right)^{Q_i}
$$
Therefore,
$$
q_{i,j} = 
\left\{\begin{array}{ll}
    \displaystyle \frac{r_i^j \lambda}{(r_i + \lambda)^{j+1}} & j < Q_i \\
    \displaystyle \left( \frac{r_i}{r_i + \lambda} \right)^j & j = Q_i \\
\end{array}\right.
$$
\section*{Size of the action space}
Let $a_d$ be the size of the action space. Then, $a_d$ is the number of ways $N$ numbers from $\{0,\frac{1}{d},\frac{2}{d},\dots,\frac{d-1}{d},1,\}$ can add to 1, i.e., the number of ways $N$ numbers from $\mathbb{N}_0$ can add to $d$. By considering the generating function of the sequence $\{a_i\}_{i\in \mathbb{N}_0}$ we get that $a_d =
\left(\begin{array}{cc}
    N+d-1\\
    d
\end{array}\right)$.
$$
\sum_{k=0}^\infty a_k x^k = (1 + x + x^2 + x^3 + \cdots)^N = (1-x)^{-N} = \sum_{k=0}^\infty \left(\begin{array}{cc}
    N+k-1\\
    k
\end{array}\right) x^k
$$

\bibliographystyle{abbrv}
\bibliography{Q-learning-bernoulli}     

\end{document}      